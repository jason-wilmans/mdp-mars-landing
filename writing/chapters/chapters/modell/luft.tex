Der komplexeste Teil des Modells beschreibt die Interaktion mit der Atmosphäre. Es werden zwei resultierende Kräfte berechnet. Die erste ist der Luftwiderstand, dem die Kapsel immer ausgesetzt ist. Zusätzlich kann, je nach Form und Angriffswinkel, dynamischer Auftrieb erzeugt werden. Die Berechnung beider Kräfte basiert auf einem Atmosphärenmodell, dass den Luftdruck in Abhängigkeit von der Höhe annähert. Von hoher Relevanz, zum einen auf Grund ihrer starken Auswirkung, zum anderen in Anbetracht der Fragestellung, sind die Flugparameter, die sich je nach Flugphase stark ändern können. Die Teilsysteme werden nun in dieser Reihenfolge beschrieben.

\paragraph{Kräfteberechnung}
Auf die detaillierte Berechnung der Strö\-mungs\-arten wurde auf Grund der schlechten Quellenlage und dem Fokus, sowie Umfang der Arbeit verzichtet. Die zugrunde liegende Komplexität drückt sich allerdings abgeschwächt in den dynamischen Parametern aus. Es gelten die allgemeinen Gleichungen für Luftwiderstand und -auftrieb setzen eine Reihe von Größen in Beziehung. Der dimensionslose Luftwiderstandsbeiwert $c_W$ beschreibt, zu welchem Anteil die Reibung durch Druck auf die dem anströmenden Gas zugerichtete Fläche verursacht wird  und wie viel durch die am Körper entlang streifende Strömung entsteht. Die relative Geschwindigkeit zum Gas $v$ ist die dominierende Größe, da sie quadratisch eingeht. Zuletzt ist der Widerstand von der Fläche abhängig. Hierbei ist zu beachten, dass die Querschnittsfläche senkrecht zur Anströmungsrichtung gemeint ist. Wölbungen auf der Achse der Anströmrichtung drücken sich nicht in der Fläche aus, sondern im $c_W$, beziehungsweise $c_A$-Wert.
\begin{equation}
	$$ F_W = c_W \cdot \frac{\rho}{2} \cdot v^{2} \cdot A $$
\end{equation}
und Auftrieb:
\begin{equation}
	$$ F_A = c_A \cdot \frac{\rho}{2} \cdot v^{2} \cdot A $$
\end{equation}
Die korrespondierenden Beschleunigungsvektoren $\vec a_W$ und $\vec a_A$ berechnen sich analog zu den bisher betrachteten Teilsystemen. Die Richtung der Kraft, die durch den Luftwiderstand ausgeübt wird, ist (wie die der Triebwerke) der Flugrichtung $\vec r$ (siehe \ref{transform}) genau entgegen gesetzt.
\begin{equation}
	$$ \vec a_W = \frac{c_W \cdot \frac{p}{2} \cdot v^{2} \cdot A}{m_K + m_T} \cdot \vec r $$
\end{equation}

Im Gegensatz dazu wirkt die Auftriebkraft genau senkrecht zur Anströmungssrichtung. Es gibt (in einem 2-dimensionalen Koordinatensystem) zwei mögliche perpendikuläre Vektoren zur Anströmungsrichtung. Wie bei der Rotation geht das Modell davon aus, dass die Fluglageregelung perfekt arbeitet. Deshalb wird angenommen, dass die Auftriebskraft, so vorhanden, immer in die Richtung jenes Vektors der beiden möglichen zeigt, der nach ''oben'' (horizontale Lage), respektive ''rechts'' (vertikale Lage) orientiert ist. Letzteres könnte ein ungewolltes Simulationsartefakt verursachen. Dieses tritt jedoch nicht auf, da die Ausftriebskraft ohnehin nur im hypersonischen Flug berücksichtigt wird. Die Richtung der Auftriebskraft $\vec l$ ist also der um 90 Grad rotierte Richtungsvektor $\vec r$. Die Gleichung für den resultierenden Beschleunigungsvektor $\vec a_A$ sieht der für den Luftwiderstand sehr ähnlich.
\begin{equation}
	$$ \vec a_A = \frac{c_A \cdot \frac{p}{2} \cdot v^{2} \cdot A}{m_K + m_T} \cdot \vec r$$
\end{equation}

\paragraph{Atmosphärenmodell}
Das verwendete Atmosphärenmodell basiert stark auf dem NASA Vorschlag \cite{Hall2015}. Es besteht aus einer Menge Zusammensetzung von Funktionen, die Daten der Mars Global Surveyor im April 1996 annähern. Es unterscheidet zwei Höhenbereiche: über 7000m und darunter. Beide Schichten sind strukturell gleich modelliert, nur unterschiedlich parametrisiert. Es gibt zwei Basisfunktionen, die Anhängig von der Höhe sind. Die lineare Funktion $T(h)$ berechnet die Temperatur ($^{\circ}$Celsius) für die gegebene Höhe $h$ (m). $p(h)$ tut dasselbe für den exponentiellen Druck (kPa). Um die daraus resultierende Dichte zu berechnen, werden diese beiden Werte in der Funktion $\rho(h)$ zusammengeführt. $p(h)$ und $\rho(h)$ sind höhenunabhängig
$$ p(h) = 0,699 \cdot e^{-0,00009 \cdot h}$$
$$ \rho(h) =  \frac{p(h)}{R \cdot (T(h) + 273,1)} $$
mit $ R = 0,1921 $. Für Höhen über 7000 definiert dass Originalmodell nun $ T(h) = -23,4 - 0,00222 \cdot h $. Für Höhen unter 7000m ändern sich die Parameter auf $ T(h) = -31 - 0,00222 \cdot h $ und $ p(h) = 0,699 \cdot e^{-0,00009 \cdot h}$.

Betrachtet man die Definition, ist zunächst die Definitionsmodell für den Wert 7000m auffällig. Dies kann recht einfach behoben werden, indem einer der Bereiche erweitert wird. Im verwendeten Modell ist die untere Schicht einschließlich 7000m definiert. Analysiert man die bestehende Funktion weiterhin \ref{atmosphereNasa}, gibt es noch drei weitere bemerkenswerte Eigenschaften.
\centerImage{atmosphereNasa}{0.3}{Landephasen}
Zum einen gibt es eine Singularität bei ca. 115km. Diese ist im Modell ebenfalls vorhanden. Aus diesem Grund wird der Output im finalen Modell zwischen 0 und 0,02 (Durchschnittswert an der Oberfläche)\cite{NASA2016} begrenzt. Das Clamping ist aber auch aus einem anderen Grund notwendig: Ab ungefähr der selben Höhe fallen die Werte unter null, was physikalisch undefiniert ist. Drittens erreicht bei im Modell der NASA die Dichte an der Oberfläche nie 0,02, was eigentlich dem Durchschnittswert entspricht.

\paragraph{dynamische Parameter}

		
In der betrachteten Literatur wird (zumindest für die hypersonische Flugphase) ein festes Verhältnis zwischen		