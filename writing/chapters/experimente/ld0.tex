\centerImage{acc0}{0.35}{$g$-Kräfte für $L/D$ = 0}
In Abbildung \ref{fig:acc0} ist die maximale, während des Fluges auftretende Kraft in Vielfachen der Erdbeschleunigung $g \simeq 9,18\frac{m}{s^2}$ für eine $L/D$ von 0 dargestellt. Das Diagramm ist dabei zu lesen wie eine Höhenkarte. An der x-Achse sind die simulierten Eintrittswinkel $\alpha$ von 10 bis $45^{\circ}$ abzulesen. Die y-Achse ist mit den unterschiedlichen Geschwindigkeiten $v$ beschriftet. Die Linien zeigen an, wie hoch der $g$-Wert für ein gegebenes Paar aus $\alpha$ und $v$ ist. Die Farben gelten für alle Diagramme gleichermaßen und beziehen sich auf die insgesamt (also über alle Werte von $L/D$ hinweg) ermittelten Minima und Maxima. Das leuchtendste Türkis steht für den kleinsten Wert ($\simeq 4,66$), das tiefste Rot für den höchsten ermittelten Wert ($\simeq 65,8$).

Zur Einordnung der Daten seien kurz einige Referenzwerte erwähnt: Fünf bis sechs $g$ führen bei Menschen gewöhnlich zur vollständigen Bewusstlosigkeit. Der höchste als Schock ohne bleibende Verletzungen zu überstehende $g-Wert$ liegt bei circa 100 \cite{Shanahan2004}. Allerdings ist immer der Zeitbezug zu berücksichtigen. Die erwähnten $5g$ können über längere Zeit wirken, ohne den Menschen nachhaltig zu beeinträchtigen. Über längere Zeit können 20$g$ zum Beispiel zum Ersticken führen.

Man beachte, dass das Modell keine durch Aufschlag auf dem Boden entstehenden Kräfte abbildet. Diese würden durch ihre Extremwerte den Erkenntnisgewinn der Diagramme deutlich reduzieren ohne Genauigkeit hinzuzufügen. Abstürze\footnote{Als Absturz wird jeder Bodenkontakt mit mehr als $2,5\frac{m}{s}$ Restgeschwindigkeit gewertet.} werden grundsätzlich nicht als sinnvolle Trajektoren betrachtet.

Betrachten wir nun die Testreihe mit diesen Informationen im Hintergrund. Grundsätzlich steigt die maximale $g$-Kraft in Abhängigkeit von Eintrittsgeschwindigkeiten und Eintrittswinkel. Links unten sind die niedrigsten Werte, rechts oben die höchsten. Dies entspricht vermutlich auch der intuitiven Erwartung. Es zeigt sich allerdings ein lokales Maximum mit dem Zentrum bei $\alpha = 15^{\circ}$ und $v = 6000\frac{m}{s}$. Die Maximalwerte liegen bei $65g$, die Minimalwerte bei 13. Ein für diesen $L/D$-Wert typischer Trajektor ist der für die Werte $\alpha = 15^{\circ}$ und $v = 6000 \frac{m}{s}$, dargestellt in Abbildung \ref{fig:trajectory-0-15-6000}.
\centerImage{trajectory-0-15-6000}{0.35}{Flugbahn für  $L/D = 0$, $\alpha = 15^{\circ}$ und $v = 6000 \frac{m}{s}$}
Charakteristisch ist eine leichte Abflachung, die erst durch das Auslösen des Fallschirms deutlich verändert wird. Der dazugehörige Zeitpunkt ist auf dem Diagramm der g-Kräfte (Abbildung \ref{fig:g-0_15-6000}) gut erkennbar. Auffällig sind auf diesem Diagramm einige Artefakte der Modellierung. Zum einen der extreme Ausschlag bei 9 Sekunden, eine Folge der Singularität im Atmosphärenmodell (\ref{par:atmosphere}). Dann der zweite, bei dem die Beschleunigung von $\simeq1,6$ auf $112g$ springt. Dieser Sprung entsteht durch das die Modellierung der dynamischen Parameter.
\centerImage{g-0_15-6000}{0.35}{$g$-Kräfte für $L/D$ = 0, $\alpha = 15^{\circ}$ und $v = 6000 \frac{m}{s}$}
Die letzte Grafik für diesen Absatz (\ref{fig:multiTrajec0}) überlagert alle Bahnen für $L/D = 0$. Die zusammenhängenden Grüppchen teilen sich jeweils den selben Eintrittswinkel.
\centerImage{multiTrajec0}{0.35}{Alle Bahnen für $L/D = 0$}