Der bei $L/D = 0,1$ beobachtete Trend setzt sich bei $L/D = 0,2$ in allen Bereichen deutlich fort. Waren für $L/D = 0,1$ erst ab Winkeln von kleiner als $25^{\circ}$ deutliche Abflachungen zu sehen, setzen diese nun schon bei $35^{\circ}$ ein. Während die minimalen zurückgelegten Entfernungen kaum schwanken, unterscheiden sich die maximalen Entfernungen um $34km$. Beim Schritt davor betrug diese Differenz $76km$. Das ist darauf zurückzuführen, dass sich nun annähernd oder zeitweise tatsächlich horizontale Flugbahnen auf einer Höhe von $7$ (hohe Winkel) bis $30km$ (kleine Winkel) ergeben. Das erlaubt es der Kapsel, länger Bewegungsenergie in Reibungskräfte umzuwandeln. Deshalb wird schneller die Grenze von Mach 2 unterschritten, ab der der Fallschirm ausgebracht werden kann.

Die Maximalbeschleunigungen entwickeln sich ebenfalls entlang der bisher beobachteten Richtung. Bei gleichem $\alpha = 22^{\circ}$ erhöht sich die mögliche Geschwindigkeit für $g > 20$ von $6055\frac{m}{s}$ auf $6217\frac{m}{s}$. Das lokale Maximum zieht sich deutlich zusammen, ohne jedoch bei diesem Schritt den Extremwert im Zentrum nennenswert zu erhöhen. Selbst für den Flug direkt im Zentrum des lokalen Maximums ($\alpha = 22^{\circ}, v = 6000\frac{m}{s}$) flacht sich die g-Kurve deutlich ab (\ref{fig:g-0_2-15-6000})

\centerImage{acc_0_2}{0.35}{$g$-Kräfte für $L/D = 0,2$}
\centerImage{g-0_2-15-6000}{0.35}{$g$ für $L/D = 0,2$, $\alpha = 15^{\circ}$ und $v = 6000 \frac{m}{s}$}
\centerImage{multiTrajec0_2}{0.35}{Alle Bahnen für $L/D = 0,2$}