Die Ergebnisse für $L/D$ sind noch stärker vom Winkel abhängig, als die bisherigen. Für steilere Winkel werden extreme Bahnen theoretisch möglich. Die ''extremste'' erfolgreiche Landung ist hierbei hier die mit den Werten $\alpha = 22^{\circ}$ und $v = 6000\frac{m}{s}$. Ob sie angesichts der extremen $g$-Entwicklung jedoch realistisch ist, bleibt fraglich. Bei unwesentlich größeren Winkeln beginnen die Bahnen, an der Atmosphäre abzuprallen. Die maximal zurückgelegte Entfernung beträgt hierbei $\simeq 1200km$. Das ist ca. ein Drittel des Mars-Radius. Bei dieser Entfernung dürfte die Krümmung der Oberfläche einen deutlichen Einfluss haben, wodurch sich Zweifel am Realismusgrad der Trajektoren für $10^{\circ}$ ergeben. Falls, wie zu vermuten, die Oberflächenkrümmung eine entscheidende Rolle spielt, müsste sich eine Form ergeben, bei der die Kapsel in mehrmals an der unteren Atmosphäre abprallt. Vergleichbar mit einem auf Wasser geflippten Stein würde sie bei jedem Eintritt Energie verlieren, bis die Geschwindigkeit zum Eintritt reicht. Solche Formen dürften extrem schwer mit Präzision zu planen sein. Unabhängig von der Qualität der Modelle gibt es eintretende unvorhersehbare Störfaktoren (insbesondere Wind), die nicht einkalkuliert werden können.

Das lokale Maximum bleibt erhalten. Die Erhöhung des $L/D$-Werts erlaubt es wieder, sich dessen Zentrum anzunähern, ohne die gewünschte g-Zahl zu überschreiten. Die Höhe des Zentrums verändert sich dabei für selbst für 6 Nachkommastellen nicht. Es entwickelte sich eine ausgeprägte Zunge mit relativ niedrigen $g$-Werten ($\leq 24,1$), deren Entwicklung sich über die gesamten Werte von $L/D$ ebenfalls gut verfolgen lässt.

\centerImage{acc_0_3}{0.35}{$g$-Kräfte für $L/D = 0,3$}
\centerImage{g-0_3-40-8000}{0.35}{$g$ für $L/D = 0,3$, $\alpha = 40^{\circ}$ und $v = 8000 \frac{m}{s}$}
\centerImage{multiTrajec0_3}{0.35}{Alle Bahnen für $L/D = 0,3$}