\label{subsec:reihen}
Für die Berechnung der Ergebnisse wurden (neben den Iterationen für die Verifikation) die zentralen Versuchsreihen durchgeführt, die verschiedene Szenarien für den Eintritt in die Atmosphäre durchspielen. Von besonderer Bedeutung sind entsprechend der Fragestellung besonders die auf die Kapsel wirkenden Kräfte und das Ende eines Simulationslaufes. Als Anfangspunkt der Simulationen wird jeweils eine Höhe von $125.000m$ angenommen, da auf dem Mars in dieser Höhe die ersten Ausläufer der Atmosphäre messbar werden.

Die variierenden Parameter der Simulation sind die Gleitzahl oder englisch Lift-To-Drag-Ratio $L/D$, der Eintrittswinkel $\alpha$ und die Eintrittsgeschwindigkeit $v$. Um die resultierenden Kräfte in Abhängigkeit von den Parametern zu testen, wurden die Parameter in gleichmäßigen Schritten miteinander kombiniert. Tabelle \ref{fig:paramCombinations} führt die verwendete Rasterisierung der jeweiligen Wertebereiche auf. In allen Kombinationen ergibt sich daraus eine Gesamtzahl von 
\autoFigure{tbp}{paramCombinations}{
\begin{tabular}{c|c c c}
	
	Parameter & Minimum & Maximum & Schrittweite \\ 
	\hline
	$L/D$ & 0 & 0.3 & 0.05 \\ 
	
	$\alpha$ & 5 & 45 & 5 \\ 
	
	$v$ & 4000 & 8000 & 1000 \\ 
	\label{tab:paramRaster}
\end{tabular}
}{Simulationsparameter} 120 Simulationen.

Die entstehenden Werte sind vierdimensional. Der für die Verifikation noch gangbare Ansatz, die Simulationsparameter in MATLAB von Hand einzutragen und die Ergebnisse mit Hilfe der Bordmittel zu untersuchen, erwies sich für diesen Schritt als nicht mehr durchführbar. Deshalb wurde das Aufrufen der Simulationsdurchläufe in einem Script automatisiert. Das hat zur Folge, dass die Betrachtung der Ergebnisse nicht mehr sinnvoll innerhalb von Simulink möglich ist. Sie werden in einer kodierten Strukur ins Dateisystem geschrieben und mit Hilfe des im nächsten Kapitel beschriebenen Simulationstools ausgewertet.