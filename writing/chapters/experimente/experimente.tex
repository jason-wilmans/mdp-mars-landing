\subsection{Versuchsreihen}
Variables:
L/D = 0.0 to 0.3 (step size: 0.05 = 5 steps)
Entry Speed = 3000 to 9000 m/s  (Stepsize 1000 m/s = 6 steps)
Entry Angle = -10$^{\circ}$ to 20$^{\circ}$ (step size: 1$^{\circ}$ = 5 steps) 

Results:
Time
Trajectory
Max/Min Air Speed and Speed on Impact
Total Horizontal Distance
Max Force

\begin{tabular}{|c|c|}
	\hline 
	&  \\ 
	\hline 
	&  \\ 
	\hline 
\end{tabular} 


\subsection{Diskussion} :Die korrektere Annäherung durch eine Ellipse zöge einige Nachteile mit sich. Zum einen ist die Berechnung von Ellipsen mathematisch ungleich komplizierter. Zusätzlich ist sie wegen der Wurzeloperation (und der vorhergehenden Rechenschritte für die Halbachsen) ungleich teurer im Rechenaufwand. Dem gegenüber steht ist nur eine Annäherung aber der ohnehin feste Angriffswinkel von $16^{\circ}$ \cite{Edquist2009}, verändert die Fläche nur unwesentlich, die Berechnung aber deutlich verkompliziert.