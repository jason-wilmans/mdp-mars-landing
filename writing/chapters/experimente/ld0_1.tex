Ab $L/D = 0,1$ verändern sich die Flugbahnen deutlich. Die sonst streng konvexen Kurven bekommen in nach Winkel und Geschwindigkeit variierender Höhe eine konkave Form. Die Extrempunkte Punkt bewegen auf einer Höhe von $60\pm 15km$. Bei kleinen Winkeln und hohen Ge\-schwin\-dig\-keit\-en beginnen sich ''Stufen'' zu bilden, bevor (in deutliche größerer Höhe als bei $L/D = 0$) der Fallschirm ausgelöst werden kann. Passend zu diesen Flugbahnen erhöhen sich die Flugzeiten und zurückgelegten Entfernungen. In Bezug auf die wirkenden Maximalbeschleunigungen bleibt die grundsätzliche Struktur erhalten. Die Fläche mit maximalen $g < 20$ verschiebt sich Richtung höherer Winkel und Geschwindigkeiten. Im Vergleich zu $L/D = 1$ bleibt nun die Kombination $\alpha = 22^{\circ}$ und $v = 6500$ unter $20g$. Gleichzeitig wird das lokale Maximum stärker. Der Wert für $\alpha = 15^{\circ}$ und $v = 6000 \frac{m}{s}$ steigt von $\simeq 37,3$ auf $\simeq 39,5$.

\centerImage{acc_0_1}{0.35}{$g$-Kräfte für $L/D$ = 0,1}
\centerImage{g-0_1-15-6000}{0.35}{$g$ für $L/D = 0,1$, $\alpha = 15^{\circ}$ und $v = 6000 \frac{m}{s}$}
\centerImage{multiTrajec0_1}{0.35}{Alle Bahnen für $L/D = 0,1$}
