\label{sec:ausblick}Das Modell könnte an vielen Stellen verbessert und ausgebaut werden. Das Auslassen der Oberflächenkrümung des Mars schränkt das besonderes sowohl für die Spanne der Gleitwerte, als auch für die Eintrittswinkel ein. Für diese Arbeit wurde der $L/D$ als konstant angenommen. Durch das Ändern des Angriffswinkels der Kapsel kann dieser allerdings variiert werden. Es bietet sich an, die Zugewinne durch einen dynamisch steuerbaren Auftrieb zu untersuchen. Dazu müsste das Modell um ''echte'' Rotation erweitert werden. Dadurch eröffnete sich auch die Möglichkeit, eine größere Spannweite von Fahrzeugen abzubilden. Besonders interessant wären Raketen mit der Möglichkeit, zur Erde zurück zu kehren.

Als etwas kleinere Weiterentwicklungen kommen unter anderem die Annahmen zum Übergang der Landephasen in Betracht. Auch könnte die dynamische Flächenberechnung deutlich genauer aufgelöst werden. Das würde die Abbildung anderer Objekte unterstützen. Als letzter großer Punkt bleibt das System der dynamischen Parameter. Es liefert zwar eine Annäherung an die echten Verhältnisse. Besser wäre aber eine dynamische Berechnung statt des tabellarischen Abspulens fester Werte.