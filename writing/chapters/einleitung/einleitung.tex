In den letzten zwei Jahrzehnten gibt es immer konkretere Bemühungen, den Mars zu besuchen. Neben den erfolgreichen Rovermissionen der NASA ist zum Zeitpunkt der Erstellung dieser Arbeit unter anderem auch ein indisches Pendant auf dem Weg zum Mars. Mehrere Länder(verbünde) planen konkret an bemannten Mars Missionen \cite{Wikipedia2016}. Spätestens, sobald eine Kolonie auf dem Mars entsteht (vermutlich aber deutlich früher) wäre es notwendig, regelmäßig Fracht Richtung Mars bringen zu können.

Die Marsatmosphäre konfrontiert Missionsplaner und Ingenieure nach wie vor mit einer großen Herausforderung. Sie ist dicht genug, um die Flugbahn eindringender Objekte massiv zu beeinflussen. Auf dem Mars herrscht ein Klima mit Jahreszeiten und starken Winden. Gleichzeitig bietet sie nicht genug Widerstand, um ausreichend zu bremsen. Deshalb ist ein Ziel, die Interaktionen von landenden Objekten mit hoher Genauigkeit vorhersagen zu können.

Diese Arbeit behandelt demgemäß die Forschungsfragen: Wie bestimmend ist der Einfluss des Gleitfaktors (englisch: Lift-To-Drag Ratio) auf Trajektorien bei Landungen auf dem Mars? Welche der entstehenden Flugbahnen sind unter der Berücksichtigung der wirkenden Kräfte realistisch? Um diese Fragestellung zu untersuchen, wurde ein Simulationsmodell der Marsatmosphäre und anderer wirkender Einflüsse implementiert.

Das nächste Kapitel beschreibt das entworfene Simulationsmodell. In Kapitel \ref{sec:experimente} werden die aktuellen Ergebnisse der durchgeführten Simulationsdurchläufe vorgestellt. Einen Ausblick auf weitere Entwicklungsmöglichkeiten liefert das Kapitel \ref{sec:ausblick}.