Die Flugsteuerung hat wiederum hat zwei erwähnenswerte Unterteilungen. Zum einen gibt es die Überwachung der Flugphasen. Diese orientiert sich stark an den Referenzmissionen. Sie ist intern als Deterministischer Endlicher Automat \centerImage{dea}{0.3}{Landephasen} modelliert. Im Unterschied zur Realität hat diese Phasenplanung keinerlei "Sicherheitsabstände" zwischen den Flugphasen. Stattdessen finden Übergänge ohne Zeitverzögerung statt (TODO: Evtl besser erklären?). Auch wurden alle Events ausgelassen, die das Gewicht oder den Schwerpunkt des Landesystems verändern.\\ \\

Der zweite Teil der Flugsteuerung ist das Controller-Setup. Sobald die "powered descent" Landephase beginnt, wird die verbleibenden Flughöhe als Signal in einen PID-Regler gespeist. In Reaktion auf dieses Signal bestimmt dieser Stärke der Triebwerke. Das Controller-Setup bildet während der "descent" Phase zusammen mit dem Flugmodell einen vollständigen Regelkreis.
