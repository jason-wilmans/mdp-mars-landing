In diesem Abschnitt wird das Simulationsmodell vorgestellt. Das Modell wurde vollständig in MATLAB/Simulink\footnote{http://de.mathworks.com/products/simulink/} realisiert. (TODO: Extension Point) Bei dieser Version des Modells handelt es sich um eine zweite Version. In der ersten wurden große Teile des Modells in so genannten MATLAB Functions\footnote{http://de.mathworks.com/help/simulink/slref/matlabfunction.html} ausgedrückt. Diese Vorgehensweise scheint nicht der bevorzugte Weg für MATLAB zu sein. Insbesondere beim automatisierten Linearisieren des Modells entstanden häufig Fehler. Die Erkenntnisse wurden in die zweite (hier vorgestellte) Version übertragen. Dieses Modell wurde von Anfang an als 2-D Version konzipiert. Insbesondere wurde versucht, wann immer möglich, Zusammenhänge über die von MATLAB angebotenen Signal Blocks auszudrücken.\\

Die Simulation ist in einer für Simulink typischen System-Subsystem-Struktur hierarchisch aufgebaut. Auf höchster Ebene unterscheiden sich die Flugsteuerung und das Flugmodell. Die Flugsteuerung hat zur Aufgabe, über die Verfügung stehenden Aktuatoren regelnden Einfluss auf den Flug zu nehmen. Hierzu überwacht es einige Kernparameter, wie die aktuelle Höhe.

Dem gegenüber steht das Flugmodell. Es modelliert die wirkenden physikalischen Kräfte und ihre Auswirkung auf wichtige Größen. Die Steuerbefehle der Flugsteuerung beeinflussen diese. Zusammen bilden die beiden Komponenten einen (indirekten) Regelkreis.