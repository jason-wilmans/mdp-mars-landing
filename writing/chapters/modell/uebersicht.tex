In diesem Abschnitt wird das Simulationsmodell vorgestellt. Das Modell ist in einer für Simulink typischen System-Subsystem-Struktur hierarchisch aufgebaut. Auf höchster Ebene unterscheiden sich die Flugsteuerung und das Flugmodell. Die Flugsteuerung hat zur Aufgabe, über die Verfügung stehenden Aktuatoren regelnden Einfluss auf den Flug zu nehmen. Hierzu überwacht es einige Kernparameter, wie die aktuelle Höhe.\\

Dem gegenüber steht das Flugmodell. Es modelliert die wirkenden physikalischen Kräfte und ihre Auswirkung auf wichtige Größen. Die Steuerbefehle der Flugsteuerung beeinflussen diese. Zusammen bilden die beiden Komponenten einen (indirekten) Regelkreis.