Die Bedeutung der Parameter wurde bereits im Absatz \ref{atmosphere} angerissen. Ihre Belegung in Abhängigkeit zum Zustand der Simulation In diesem Abschnitt wird beschrieben, wie sich diese während der Simulation entwickeln. Die Tabelle \ref{dynParamTable} zeigt eine Übersicht der Belegungen in Abhängigkeit zu den Variablen $v = $ Geschwindigkeit in Mach, $f =$ Fallschirm aktiviert, sowie $i =$ Fallschirm ist intakt. Im Folgenden werden die getroffenen Annahmen und Quellen für diese beschrieben.

\autoFigure{tbp}{tab:dynParamTable}{
	\begin{tabular}{c|c|c|c}

		Bedinungen & $A$ & $c_W$ & $c_A$ \\
		\hline 
		$v > 2$ & $ 15,9m^2$ & $1,68$ & $c_W \cdot L/D$ \\

		$v \leq 2 \land f \land i$ & $189,79m^2$ & $1,33$ & 0 \\
		
		$v \leq 2 \land  \lnot f \land i$ & $ 15,9m^2$ & $0,34$ & 0 \\
		
		$v \leq 2 \land f \land \lnot i$ & $ 15,9m^2$ & $0,35$ & 0 \\

		$v \leq 2 \land \lnot f \land \lnot i$ & $ 15,9m^2$ & $0,34$ & 0 \\
	\end{tabular} 
}{Belegung der Parameter nach Zustand}

Der Strömungswiderstandskoeffizient $c_W$ ist in erster Linie von der Form des beschriebenen Gegenstands abhängig, aber auch von der herrschenden Art der Luftreibung. Diese wiederum hängt stark mit der Geschwindigkeit, der herrschenden Temperatur und dem Druck zusammen. Diese Zusammenhänge sind nur in einer groben Auflösung im Simulationsmodell abgebildet. Die meisten Teilaspekte sind eigene Forschungsthemen für sich (\cite{Blanchard1980}, \cite{Edquist2009}, \cite{Theisinger2009}, \cite{Yamada2009}). Ihre Lösungen müssen häufig mit Finite-Elemente-Simulationen angenähert werden \cite{Edquist2009}. In diesem Modell wird nicht versucht, diese Untersuchungen nachzustellen. Allerdings werden ihre Ergebnisse berücksichtigt. \cite{Wells2000} als zuverlässigste Quelle arbeitet mit der Annahme, der hypersonische $c_W$ für eine Aeroshell wie beim MSL liege bei konstant bei $1,68$. Zu den anderen Phasen, insbesondere in der Nähe der Schallmauer, gab es leider keine Angaben. Deswegen wird der Koeffizient für Geschwindigkeiten unter Mach 2 mit einer konvexen Halbkugelschale approximiert ($c_W = 0,34$). Es gibt jedoch eine Ausnahme. Wurde der Fallschirm vorher bei einer Geschwindigkeit über Mach 2.2 \cite{Way2007} \cite{Edquist2009} entfaltet, wird er beschädigt und bremst nun nur noch minimal. In diesem Fall ist der $c_W = 0,35$. Wenn der Fallschirm korrekt entfaltet ist, wird der $c_W$-Wert des gesamten des Fallschirm-Kapsel-Gespanns als 1,33 angenommen. Dies entspricht der konkaven Seite einer Halbkugel und kommt im subsonischen Bereich der Realität ziemlich nahe. Zu den super- und transsonischen Eigenschaften sind den Autor keine Quellen bekannt. Im Modell ausgelassen sind damit einige Spezialeffekte, die vorstellbar wären, wie z.B. der Einfluss von Verwirbelungen der Kapsel auf den Luftstrom, der auf den Fallschirm trifft.

Die Gründe, warum sich die Fläche verändert, sind ähnlich denen beim Widerstandskoeffizienten. Im hypersonischen Flug wird ausschließlich die Kapsel umströmt. Der einzige Unterschied ist, dass die Flääche sich in großer Überschallgeschwindigkeit nicht ändert. Es wird jederzeit von einem perfekt kreisförmigen Querschnitt ausgegangen. Deshalb ist in diesem Fall $ A = \pi \cdot r^2 \sim 15,9m^2$ für den MSL Aeroshell Radius von $2,25m$ \cite{Edquist2009}. Ist der Fallschirm aktiv und intakt, wird die Fläche mit Hilfe des Radius des MSL Vorbilds $ r = 7,78m $ \cite{NASA/JPL2009} auf circa $\sim 189,79m^2$ berechnet. Ist der Fallschirm hingegen zerstört, bleibt als Fläche die "normale" Kapselfläche.

Die letzte Größe mit Abhängigkeit zur Flugphase ist der Auftriebskoeffizient $c_A$. In der betrachteten Literatur wird (für die hypersonische Flugphase) nur die lift-to-drag ratio $L/D = \frac{c_A}{c_W}$ angegeben. Aus diesem Verhältnis ergibt sich $c_A = L/D \cdot c_W$. Für die MSL Aeroshell beträgt der $L/D$ 0,24 \cite{Way2007}, \cite{Edquist2009}. Zum $c_A$ bei trans- oder subsonischem Flug waren keine Quellen auffindbar. Aus diesem Grund ist $c_A$ für Geschwindigkeiten < Mach 2 grundsätzlich auf 0 festgelegt, obwohl theoretisch bei Unterschallgeschwindigkeit deutlich größere $L/D$ erreichbar sind, als im hypersonischen Bereich.