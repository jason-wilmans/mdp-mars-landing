Dieses Kapitel beschreibt das verwendete Simulationsmodell im Detail. Der Abschnitt \ref{subsec:assumptions} erwähnt allgemeine Annahmen für das Modell. Dazu gehören Missionen, die als Referenzen dienten. Nebenbei werden einige häufig genutzte Begriffe und Abkürzungen eingeführt. Die Abschnitte \ref{subsec:parts}, \ref{subsec:controller} und \ref{subsec:model} beschreiben das eigentliche Simulationsmodell. Im letzten Abschnitt (\ref{subsec:verification}) wird der Verfifkationsprozess angeschnitten.

\subsection{Annahmen}
\label{subsec:assumptions}
\paragraph{Referenzmissionen}
Die UdSSR 
Es gab schon erfolgreiche Missionen, die als Vorlage dienen. Hauptbezug: Curiosity/Mars Sciene Laboratory

\paragraph{Landegefährt}
Um besonders für den Verfikationsprozess möglichst vergleichbar zu sein, kopiert die angenommene Landekapsel im wesentlichen das Design des EDL Systems der Mars Science Laboratory Mission. Sie ist beladen genauso schwer wie das Original, hat eine vergleichbare Stärke der Bremsraketen, führt die exakt gleiche Menge Treibstoff mit sich und benutzt den selben Fallschirm. Anders als bei der Curiosity-Mission (MSL) wird allerdings die ganze Kapsel auf dem Boden aufgesetzt. Dies scheint grundsätzlich auch für eine Kapsel der Größe des MSLs möglich zu sein, da vorherige Missionen mit kleineren Kapseln so verfuhren. Auch gibt die NASA als Hauptgrund für das komplizierte ''Sky-Crane''-Manöver das Vermeiden des Aufwirbelns von Staub an\footnote{https://www.youtube.com/watch?v=h2I8AoB1xgU}.

\paragraph{Allgemeine Annahmen}
Das Koordinatensystem des Modells ist 2-dimensional. Das ist notwendig, da eindimensionale Berechnungen zu fernab der Realität sind. Insbesondere der Eintrittswinkel spielt eine entscheidende Rolle. Übersteigt dieser deutlich $20^{\circ}$, was den eindimensionalen Fall annähert, ist die zurückgelegte Strecke in nennenswerter Atmosphäre grundsätzlich zu kurz, um die üblichen Geschwindigkeiten abzubremsen. Dies wurde in früheren, eindimensionalen, Versionen des Modells recht deutlich.
Das Modell nimmt darüber hinaus die Oberfläche des Mars als flach an. Der Vektor $(0 \,, 1)^T$ beschreibt per Konvention die Richtung vom Marsmittelpunkt nach oben.

\subsection{Übersicht der Modellteile}
\label{subsec:parts}
In diesem Abschnitt wird das Simulationsmodell vorgestellt. Das Modell wurde vollständig in MATLAB/Simulink\footnote{http://de.mathworks.com/products/simulink/} realisiert. (TODO: Extension Point) Bei dieser Version des Modells handelt es sich um eine zweite Version. In der ersten wurden große Teile des Modells in so genannten MATLAB Functions\footnote{http://de.mathworks.com/help/simulink/slref/matlabfunction.html} ausgedrückt. Diese Vorgehensweise scheint nicht der bevorzugte Weg für MATLAB zu sein. Insbesondere beim automatisierten Linearisieren des Modells entstanden häufig Fehler. Die Erkenntnisse wurden in die zweite (hier vorgestellte) Version übertragen. Dieses Modell wurde von Anfang an als 2-D Version konzipiert. Insbesondere wurde versucht, wann immer möglich, Zusammenhänge über die von MATLAB angebotenen Signal Blocks auszudrücken.\\

Die Simulation ist in einer für Simulink typischen System-Subsystem-Struktur hierarchisch aufgebaut. Auf höchster Ebene unterscheiden sich die Flugsteuerung und das Flugmodell. Die Flugsteuerung hat zur Aufgabe, über die Verfügung stehenden Aktuatoren regelnden Einfluss auf den Flug zu nehmen. Hierzu überwacht es einige Kernparameter, wie die aktuelle Höhe.

Dem gegenüber steht das Flugmodell. Es modelliert die wirkenden physikalischen Kräfte und ihre Auswirkung auf wichtige Größen. Die Steuerbefehle der Flugsteuerung beeinflussen diese. Zusammen bilden die beiden Komponenten einen (indirekten) Regelkreis.
	
\subsection{Flugsteuerung}
\label{subsec:controller}
Die Flugsteuerung hat wiederum hat zwei erwähnenswerte Unterteilungen. Zum einen gibt es die Überwachung der Flugphasen. Diese orientiert sich stark an den Referenzmissionen. Sie ist intern als Deterministischer Endlicher Automat \centerImage{dea}{0.3}{Landephasen} modelliert. Im Unterschied zur Realität hat diese Phasenplanung keinerlei "Sicherheitsabstände" zwischen den Flugphasen. Stattdessen finden Übergänge ohne Zeitverzögerung statt (TODO: Evtl besser erklären?). Auch wurden alle Events ausgelassen, die das Gewicht oder den Schwerpunkt des Landesystems verändern.\\ \\

Der zweite Teil der Flugsteuerung ist das Controller-Setup. Sobald die "powered descent" Landephase beginnt, wird die verbleibenden Flughöhe als Signal in einen PID-Regler gespeist. In Reaktion auf dieses Signal bestimmt dieser Stärke der Triebwerke. Das Controller-Setup bildet während der "descent" Phase zusammen mit dem Flugmodell einen vollständigen Regelkreis.

	
\subsection{Flugmodell}
\label{subsec:model}
\paragraph{Flugmodell}
Das Flugmodell berechnet die tatsächliche Physik des Fluges. Es besteht aus vier Teilsystemen, welche nun im Detail vorgestellt werden: Transform, Schwerkraft, Luftwiderstand und Antrieb. \\ \\

Das Transform hat seinen Namen in Anlehnung an viele Vektor-Bibliotheken in der Informatik. Es verwaltet die drei Orientierungsvariablen Geschwindigkeit, Position und Rotation. Das System hat als Parameter den aktuellen Beschleunigungsvektor. Die Beschleunigung entspricht der Summe der Einzelbeschleunigungen der anderen vier Blöcke. Die Geschwindigkeit ist das Integral der Beschleunigung, die Position integriert entsprechend die Geschwindigkeit.

Die Rotation hingegen ist nicht als unabhängige dynamische Größe modelliert. Sie wird als ideal gesteuert, i.e. immer der aktuellen Tangente des Flugtrajektors entgegengesetzt, angenommen. Entsprechend wird die Rotation als ein Vektor $\vec r$ dynamisch aus dem Geschwindigkeitsvektor $\vec v$ berechnet als $\vec r = -\frac{\vec v}{|\vec v|}$.\\ \\

Die Gravitation berechnet sich entsprechend der allgemeinen Formel für die Gravitationskraft.
$$F = G \frac{m_1 \cdot m_2}{r^{2}} $$
Sie berücksichtigt die Masse des Mars $m_M = 5.9724 \cdot 10^{24}$kg \cite{NASA2016}, die Masse der Landekapsel $m_K = 2401$kg \cite{Wikipedia2016b} und die dynamische Masse des Treibstoffs $m_T$, die zwischen 0 und 390 kg \cite{Wikipedia2016b} liegen kann. Mit $F = m \cdot a$ berechnet sich der Beschleunigungsvektors $\vec a_G$ wie folgt.
$$\vec a_G = G \frac{m_M \cdot (m_K + m_T)}{r^{2}} \cdot \frac{1}{m_K + m_T} \cdot \left(\begin{array}{c} 0 \\ -1 \end{array}\right)$$\\

Weitere Beschleunigung erfährt die Kapsel potentiell durch das Antriebssystem. Antriebssysteme definieren sich direkt durch ihre Schubkraft $F_T$. Diese ist laut Herstellerangaben zusammengerechnet 24.8

	Luftwiderstand
		NASA Atmosphärenmodell (Anpassungen, Singularität!)
			Luftwiderstand
			Auftriebskraft
		dynamische Parameter
Antrieb
	Trägheitssimulation
	Treibstoffverbrauch

\subsection{Verifikation}
\label{subsec:verification}
Das Simulationsmodell wurde mehrfach auf Mikro- wie auf Makroebene auf Korrektheit überprüft. Einige der Verifikationskriterien werden hier beschrieben. Auf Mikroebene kann zunächst das Problem auf das Prüfen der Korrektheit einzelner Systemteile herunter gebrochen werden. Entsprechend wurden die Einzelsysteme in verschiedenen Simulationsläufen getrennt untersucht. Das Atmosphärenmodell wurde direkt von der NASA übernommen. Zu prüfen ist hier also mehr die Korrektheit als. Zu diesem Zweck wurde dessen Berechnungen unabhängig vom Simulationsmodell geplottet (s. \ref{par:atmosphere}) und mit real gemessenen Daten \cite{Blanchard1980} verglichen. Um die anderen Systemteile zu testen, wurden die jeweils anderen Modellteile abgeschaltet, so dass die Wirkung der Einzelsysteme sichtbar wird. Durch provozieren bestimmter Situation, deren Verlauf bekannt ist, wurde außerdem das Vorhandensein bestimmter gewollter Artefakte überprüft. Ein Beispiel für so ein gewolltes Artefakt ist der plötzliche Absturz bei einem Landeversuch, wenn der Treibstoff vor der Landung zuneige geht.

Wie erwähnt wurden fast alle Parameter der simulierten Mission gleich denen des MSL gehalten. Dies hat zur Folge, dass man direkte Vergleiche zu den Aufzeichnungen und Simulationen der NASA ziehen kann. Die Untersuchung auf Makroebene spricht für eine hohe Validität des Modells. Beispielsweise ein Vergleich der Höhe in Abhängigkeit von der Geschwindigkeit im Original (\ref{fig:speedToHeight_original}) verglichen mit dem Simulationsergebnis ((\ref{fig:speedToHeight_msl})
\centerImage{speedToHeight_original}{0.18}{Höhe nach Geschwindigkeit mit MSL Parametern}
\centerImage{speedToHeight_msl}{0.3}{Höhe nach Geschwindigkeit mit MSL Parametern}