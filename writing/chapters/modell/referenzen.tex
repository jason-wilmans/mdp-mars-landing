Seit 1960 gab es zahlreiche erfolgreiche (und fehlgeschlagene) Missionen zum Mars und seinen Satelliten. Der Rechercheaufwand für genaue Details einzelner Missionen ist sehr hoch, so dass im Rahmen dieser Arbeit eine Begrenzung vorgenommen werden muss. Die NASA Lander Missionen seit Viking 1 (1975) sind am zugänglichsten dokumentiert und weisen die höchste Erfolgsquote auf. Zudem haben sie auf einander aufbauende Landeprozeduren. Deswegen dienen sie als Hauptreferenzen für dieses Modell die letzten beiden Missionen, besonders die letzte Mars Exploration Rover (MER) Mission, Mars Science Laboratory, das erfolgreich den knapp eine Tonne schweren Rover Curiosity absetzte. Dies ist besonders wegen der bisher größten Nutzlast die interessanteste Mission.

 Angriffswinkel beschreibt in diesem Zusammenhang den Winkel im Vergleich zur Anströmrichtung der Atmosphäre. (TODO: Beschreiben. Maybe Diagramm?)
 (TODO: angenommenes Fluggerät beschreiben)
 (TODO: Phase beschreiben)