Seit 1960 gab es zahlreiche erfolgreiche (und fehlgeschlagene) Missionen zum Mars und seinen Satelliten. Der Rechercheaufwand für genaue Details einzelner Missionen ist sehr hoch, so dass im Rahmen dieser Arbeit eine Begrenzung vorgenommen werden muss. Die NASA Lander Missionen seit Viking 1 (1975) sind am zugänglichsten dokumentiert und weisen die höchste Erfolgsquote auf. Zudem haben sie auf einander aufbauende Landeprozeduren. Aus diesen Gründen dienen sie als Hauptreferenzen für dieses Modell. Das Hauptaugenmerk liegt auf der letzten Mars Exploration Rover (MER) Mission, Mars Science Laboratory, die erfolgreich den knapp eine Tonne \cite{Way2007} schweren Rover Curiosity absetzte. Diese ist besonders wegen der bisher größten Nutzlast die interessanteste Mission.

Das für alle Missionen einheitliche Schema für die Landungen wird Entry, Descent, Landing (EDL) genannt. Der Begriff unterscheidet drei Phasen während eines Landevorgangs, die alle bisherigen Missionen teilen. In der Entry-Phase dringt eine Aeroshell genannte Kapsel in die obere Marsatmosphäre ein. Sie benötigt dafür einen hoch effektiven Hitzeschild, da die kinetische Energie der Kapsel durch Luftreibung fast vollständig \footnote{MSL: $>99\%$ \cite{Edquist2009}} in Hitze umgewandelt wird. Je nach Eintrittswinkel und Geschwindigkeit hat die Kapsel dabei einen Angriffswinkel relativ zur Anströmrichtung (TODO: Beschreiben. Maybe Diagramm?). Dieser ermöglicht es, Auftrieb zu erzeugen. Bei der MSL-Landung wurde zum ersten Mal die Fluglage aktiv kontrolliert, um spontan auf unplanbare Einflüsse, insbesondere auf Wind, reagieren zu können.
Sobald die Geschwindigkeit auf supersonische Größen gefallen ist, wird ein Fallschirm ausgeworfen. Dieser reduziert die Fallgeschwindigkeit nochmal deutlich und lenkt die Flugrichtung Richtung Boden. Die Landing-Phase ist bei den Missionen sehr unterschiedlich. Da der Fallschirm nicht ausreichend bremst, müssen Restgeschwindigkeiten in der Größenordnung von $100m/s$ abgebremst werden. Hierbei spielten bisher immer Bremsraketen eine Rolle. Im Detail unterscheiden sich die letzten Schritte. Zum Beispiel wurde bei der Pathfinder Mission nach Nutzung der Bremsraketen ein mit Airbags geschützter ''Ball'' fallen gelassen. Im Vergleich dazustabilisierte sich bei der MSL Landung ein ''Sky Crane'' und ließ den Rover an Seilen herunter.