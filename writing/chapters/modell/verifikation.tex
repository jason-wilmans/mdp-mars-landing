Das Simulationsmodell wurde mehrfach auf Mikro- wie auf Makroebene auf Korrektheit überprüft. Einige der Verifikationskriterien werden hier beschrieben. Auf Mikroebene kann zunächst das Problem auf das Prüfen der Korrektheit einzelner Systemteile herunter gebrochen werden. Entsprechend wurden die Einzelsysteme in verschiedenen Simulationsläufen getrennt untersucht. Das Atmosphärenmodell wurde direkt von der NASA übernommen. Zu prüfen ist hier also mehr die Korrektheit als. Zu diesem Zweck wurde dessen Berechnungen unabhängig vom Simulationsmodell geplottet (s. \ref{par:atmosphere}) und mit real gemessenen Daten \cite{Blanchard1980} verglichen. Um die anderen Systemteile zu testen, wurden die jeweils anderen Modellteile abgeschaltet, so dass die Wirkung der Einzelsysteme sichtbar wird. Durch provozieren bestimmter Situation, deren Verlauf bekannt ist, wurde außerdem das Vorhandensein bestimmter gewollter Artefakte überprüft. Ein Beispiel für so ein gewolltes Artefakt ist der plötzliche Absturz bei einem Landeversuch, wenn der Treibstoff vor der Landung zuneige geht.

Wie erwähnt wurden fast alle Parameter der simulierten Mission gleich denen des MSL gehalten. Dies hat zur Folge, dass man direkte Vergleiche zu den Aufzeichnungen und Simulationen der NASA ziehen kann. Die Untersuchung auf Makroebene spricht für eine hohe Validität des Modells. Beispielsweise ein Vergleich der Höhe in Abhängigkeit von der Geschwindigkeit im Original (\ref{fig:speedToHeight_original}) verglichen mit dem Simulationsergebnis ((\ref{fig:speedToHeight_msl})
\centerImage{speedToHeight_original}{0.18}{Höhe nach Geschwindigkeit mit MSL Parametern}
\centerImage{speedToHeight_msl}{0.3}{Höhe nach Geschwindigkeit mit MSL Parametern}