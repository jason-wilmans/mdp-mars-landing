Das verwendete Atmosphärenmodell basiert stark auf dem NASA Vorschlag \cite{Hall2015}. Es besteht aus einer Menge Zusammensetzung von Funktionen, die Daten der Mars Global Surveyor im April 1996 annähern. Es unterscheidet zwei Höhenbereiche: über 7000m und darunter. Beide Schichten sind strukturell gleich modelliert, nur unterschiedlich parametrisiert. Es gibt zwei Basisfunktionen, die Anhängig von der Höhe sind. Die lineare Funktion $T(h)$ berechnet die Temperatur ($^{\circ}$Celsius) für die gegebene Höhe $h$ (m). $p(h)$ tut dasselbe für den exponentiellen Druck (kPa). Um die daraus resultierende Dichte zu berechnen, werden diese beiden Werte in der Funktion $\rho(h)$ zusammengeführt. $p(h)$ und $\rho(h)$ sind höhenunabhängig
$$ p(h) = 0,699 \cdot e^{-0,00009 \cdot h}$$
$$ \rho(h) =  \frac{p(h)}{R \cdot (T(h) + 273,1)} $$
mit $ R = 0,1921 $. Für Höhen über 7000 definiert dass Originalmodell nun $ T(h) = -23,4 - 0,00222 \cdot h $. Für Höhen unter 7000m ändern sich die Parameter auf $ T(h) = -31 - 0,00222 \cdot h $ und $ p(h) = 0,699 \cdot e^{-0,00009 \cdot h}$.

Betrachtet man die Definition, ist zunächst die Definitionsmodell für den Wert 7000m auffällig. Dies kann recht einfach behoben werden, indem einer der Bereiche erweitert wird. Im verwendeten Modell ist die untere Schicht einschließlich 7000m definiert. Analysiert man die bestehende Funktion weiterhin \ref{atmosphereNasa}, gibt es noch drei weitere bemerkenswerte Eigenschaften.
\centerImage{atmosphereNasa}{0.3}{Landephasen}
Zum einen gibt es eine Singularität bei ca. 115km. Diese ist im Modell ebenfalls vorhanden. Aus diesem Grund wird der Output im finalen Modell zwischen 0 und 0,02 (Durchschnittswert an der Oberfläche)\cite{NASA2016} begrenzt. Das Clamping ist aber auch aus einem anderen Grund notwendig: Ab ungefähr der selben Höhe fallen die Werte unter null, was physikalisch undefiniert ist. Drittens erreicht bei im Modell der NASA die Dichte an der Oberfläche nie 0,02, was eigentlich dem Durchschnittswert entspricht.