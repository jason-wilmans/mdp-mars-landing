Das verwendete Atmosphärenmodell basiert stark auf dem NASA Vorschlag \cite{Hall2015}. Es besteht aus einer Kombination von Funktionen, die Daten der Mars Global Surveyor im April 1996 annähern. Es unterscheidet zwei Höhenbereiche: über $7000m$ und darunter. Beide Schichten sind strukturell gleich modelliert, nur unterschiedlich parametrisiert. Es gibt zwei Basisfunktionen, die abhängig von der Höhe sind. Die lineare Funktion $T(h)$ berechnet die Temperatur ($^{\circ}$Celsius) für die gegebene Höhe $h$ (m). $p(h)$ tut dasselbe für den exponentiellen Druck (kPa). Um die daraus resultierende Dichte zu berechnen, werden diese beiden Werte in der Funktion $\rho(h)$ zusammengeführt. $p(h)$ und $\rho(h)$ sind höhenunabhängig
\begin{equation}
	$$ p(h) = 0,699 \cdot e^{-0,00009 \cdot h}$$
\end{equation}
\begin{equation}
	$$ \rho(h) =  \frac{p(h)}{R \cdot (T(h) + 273,1)} $$
\end{equation}
mit $ R = 0,1921 $. Für Höhen über $7000m$ definiert dass Originalmodell nun $ T(h) = -23,4 - 0,00222 \cdot h $. Für Höhen unter $7000m$ ändern sich die Parameter auf $ T(h) = -31 - 0,00222 \cdot h $ und $ p(h) = 0,699 \cdot e^{-0,00009 \cdot h}$.

Betrachtet man die Definition, ist zunächst die Definitionslücke für den Wert $7000m$ auffällig. Dies kann recht einfach behoben werden, indem einer der Bereiche erweitert wird. Im finalen Modell ist die untere Schicht einschließlich $7000m$ definiert. Analysiert man die bestehende Funktion weiterhin (Abbildung \ref{fig:atmosphere}), zeigen sich zwei weitere bemerkenswerte Eigenschaften.
\centerImage{atmosphere}{0.3}{Dichte nach Höhe}
Zum einen gibt es eine Singularität beim Wert $112550m$ herum. Aus diesem Grund wird der Output im finalen Modell zwischen null und $0,02$ (Durchschnittswert an der Oberfläche)\cite{NASA2016} begrenzt. Das Clamping ist aber auch aus einem anderen Grund notwendig: Ab ungefähr der selben Höhe fällt die angegebene Dichte unter null, was physikalisch undefiniert ist. Anscheinend wurde das Modell nicht für Höhen über $110 km$ entworfen. Die berechneten Temperaturwerte, die im Atmosphärenmodell als Zwischenergebnisse anfallen, dienen der Flugsteuerung außerdem zur Berechnung der Schallgeschwindigkeit.