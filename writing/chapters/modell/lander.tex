Um besonders für den Verfikationsprozess möglichst vergleichbar zu sein, kopiert die angenommene Landekapsel im wesentlichen das Design des EDL Systems der Mars Science Laboratory Mission. Sie ist beladen genauso schwer wie das Original, hat eine vergleichbare Stärke der Bremsraketen, führt die exakt gleiche Menge Treibstoff mit sich und benutzt den selben Fallschirm. Anders als bei der Curiosity-Mission (MSL) wird allerdings die ganze Kapsel auf dem Boden aufgesetzt. Dies scheint grundsätzlich auch für eine Kapsel der Größe des MSLs möglich zu sein, da vorherige Missionen mit kleineren Kapseln so verfuhren. Auch gibt die NASA als Hauptgrund für das komplizierte ''Sky-Crane''-Manöver das Vermeiden des Aufwirbelns von Staub an\footnote{https://www.youtube.com/watch?v=h2I8AoB1xgU}.