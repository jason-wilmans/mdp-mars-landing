Auf die detaillierte Berechnung der Strömungsarten wurde auf Grund der schlechten Quellenlage und dem Fokus, sowie Umfang der Arbeit verzichtet. Die zugrunde liegende Komplexität drückt sich allerdings abgeschwächt in den dynamischen Parametern aus. Es gelten die allgemeinen Gleichungen für Luftwiderstand und -auftrieb setzen eine Reihe von Größen in Beziehung. Der dimensionslose Luftwiderstandsbeiwert $c_W$ beschreibt, zu welchem Anteil die Reibung durch Druck auf die dem anströmenden Gas zugerichtete Fläche verursacht wird  und wie viel durch die am Körper entlang streifende Strömung entsteht. (TODO: Beispielmit Platte im Wind?). Die relative Geschwindigkeit zum Gas $v$ ist die dominierende Größe, da sie quadratisch eingeht. Zuletzt ist der Widerstand von der Fläche abhängig. Hierbei ist zu beachten, dass die Querschnittsfläche senkrecht zur Anströmungsrichtung gemeint ist. Wölbungen auf der Achse der Anströmrichtung drücken sich nicht in der Fläche aus, sondern im $c_W$oder $c_A$-Wert.
$$ F_W = c_W \cdot \frac{\rho}{2} \cdot v^{2} \cdot A $$
und Auftrieb:
$$ F_A = c_A \cdot \frac{\rho}{2} \cdot v^{2} \cdot A $$
Die korrespondierenden Beschleunigungsvektoren $\vec a_W$ und $\vec a_A$ berechnen sich analog zu den bisher betrachteten Teilsystemen. Die Richtung der Kraft, die durch den Luftwiderstand ausgeübt wird, ist (wie die der Triebwerke) der Flugrichtung $\vec r$ (siehe \ref{transform}) genau entgegen gesetzt.
$$ \vec a_W = \frac{c_W \cdot \frac{p}{2} \cdot v^{2} \cdot A}{m_K + m_T} \cdot \vec r $$

Im Gegensatz dazu wirkt die Auftriebkraft genau senkrecht zur Anströmungssrichtung. Es gibt (in einem 2-dimensionalen Koordinatensystem) zwei mögliche perpendikuläre Vektoren zur Anströmungsrichtung. Wie bei der Rotation geht das Modell davon aus, dass die Fluglageregelung perfekt arbeitet. Deshalb wird angenommen, dass die Auftriebskraft, so vorhanden, immer in die Richtung jenes Vektors der beiden möglichen zeigt, der nach ''oben'' (horizontale Lage), respektive ''rechts'' (vertikale Lage) orientiert ist. Letzteres könnte ein ungewolltes Simulationsartefakt verursachen. Dieses tritt jedoch nicht auf, da die Ausftriebskraft ohnehin nur im hypersonischen Flug berücksichtigt wird. Die Richtung der Auftriebskraft $\vec l$ ist also der um 90 Grad (TODO: Berechnung der Rotation dokumentieren) rotierte Richtungsvektor $\vec r$. Die Gleichung für den resultierenden Beschleunigungsvektor $\vec a_A$ sieht der für den Luftwiderstand sehr ähnlich.
$$ \vec a_A = \frac{c_A \cdot \frac{p}{2} \cdot v^{2} \cdot A}{m_K + m_T} \cdot \vec r$$