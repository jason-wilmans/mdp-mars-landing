Die Gravitation berechnet sich entsprechend der allgemeinen Formel für die Gravitationskraft.
$$F = G \frac{m_1 \cdot m_2}{r^{2}} $$
Sie berücksichtigt die Masse des Mars $m_M = 5.9724 \cdot 10^{24}$kg \cite{NASA2016}, die Masse der Landekapsel $m_K = 2401$kg \cite{Wikipedia2016b} und die dynamische Masse des Treibstoffs $m_T$, die zwischen 0 und 390 kg \cite{Wikipedia2016b} liegen kann. Mit $F = m \cdot a$ berechnet sich der Beschleunigungsvektors $\vec a_G$ wie folgt.
$$\vec a_G = G \frac{m_M \cdot (m_K + m_T)}{r^{2}} \cdot \frac{1}{m_K + m_T} \cdot \left(\begin{array}{c} 0 \\ -1 \end{array}\right)$$\\