Weitere Beschleunigung erfährt die Kapsel potentiell durch das Antriebssystem. Antriebssysteme definieren sich direkt durch ihre Schubkraft $F_T$. Diese ist laut Herstellerangaben zusammengerechnet 24.8 kN \cite{AerojetRocketdyne2012} \cite{AerojetRocketdyne}. Die auf Bildern angedeutete Neigung der verschiedenen Triebwerke am MSL ist hierbei abstrahiert. Auch die Positionierungen der Düsen (und damit die potentiell entstehenden Momente) wurden zugunsten einer niedrigeren Komplexität zu einem gebündelten Strahl, der im Schwerpunkt greift, vereinfacht. So ist die Beschleunigung mit dynamischem Gewicht  $ a_T = \dfrac{F_T}{m_K + m_T}$.

Allerdings muss die Stärke für die Flugsteuerung einstellbar sein. Um dies zu berücksichtigen, wird die Kraft mit dem Parameter $t \in \{t | t \in \mathbb{R} \land 0 \geq t \leq 1\}$ multipliziert. Um den Realismus des Triebwerkes deutlich zu erhöhen, wird der Parameter $t$ allerdings nicht direkt benutzt. Das Signal der Flugsteuerung wird stattdessen um 200 ms verzögert und simuliert mit Hilfe einer Transfer Function (TODO: Parametrisierung, außerdem: erklären? ) ein Sättigungsverhalten. Notiert man die Verzögerung $d$(elay) und die Sättigung $s$(aturation) als Funktionen auf der gewünschten Leistung, ergibt sich die Gleichung des Beschleunigungsvektors. $\vec r$ ist dabei der unter \ref{transform} beschriebene, normalisierte, Richtungsvektor der Kapsel.
$$\vec a_{T} = \frac{d(s(t))F_T}{m_K + m_T} \cdot \vec r$$\\