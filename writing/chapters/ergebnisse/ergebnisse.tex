\label{subsec:results}
Dem Autor ist kein Weg bekannt, die Datensätze in vierdimensionaler Gänze vollständig aggregiert darzustellen. Deshalb wurden sie entlang der unterschiedlichen $L/D$ Werte aufgespalten. Dieser Abschnitt orientiert sich an den dadurch entstehenden, dreidimensionalen Paketen. Es folgt eine Betrachtung der Entwicklung der Ergebnisse für die $L/D$-Werte $0$, $0,1$, $0,2$, und $0,3$.

\paragraph{$L/D = 0$}
\centerImage{acc0}{0.35}{$g$-Kräfte für $L/D$ = 0}
In Abbildung \ref{fig:acc0} ist die maximale, während des Fluges auftretende Kraft in Vielfachen der Erdbeschleunigung $g \simeq 9,18\frac{m}{s^2}$ für eine $L/D$ von 0 dargestellt. Das Diagramm ist dabei zu lesen wie eine Höhenkarte. An der x-Achse sind die simulierten Eintrittswinkel $\alpha$ von 10 bis $45^{\circ}$ abzulesen. Die y-Achse ist mit den unterschiedlichen Geschwindigkeiten $v$ beschriftet. Die Linien zeigen an, wie hoch der $g$-Wert für ein gegebenes Paar aus $\alpha$ und $v$ ist. Die Farben gelten für alle Diagramme gleichermaßen und beziehen sich auf die insgesamt (also über alle Werte von $L/D$ hinweg) ermittelten Minima und Maxima. Das leuchtendste Türkis steht für den kleinsten Wert ($\simeq 4,66$), das tiefste Rot für den höchsten ermittelten Wert ($\simeq 65,8$).

Zur Einordnung der Daten seien kurz einige Referenzwerte erwähnt: Fünf bis sechs $g$ führen bei Menschen gewöhnlich zur vollständigen Bewusstlosigkeit. Der höchste als Schock ohne bleibende Verletzungen zu überstehende $g-Wert$ liegt bei circa 100 \cite{Shanahan2004}. Allerdings ist immer der Zeitbezug zu berücksichtigen. Die erwähnten $5g$ können über längere Zeit wirken, ohne den Menschen nachhaltig zu beeinträchtigen. Über längere Zeit können 20$g$ zum Beispiel zum Ersticken führen.

Man beachte, dass das Modell keine durch Aufschlag auf dem Boden entstehenden Kräfte abbildet. Diese würden durch ihre Extremwerte den Erkenntnisgewinn der Diagramme deutlich reduzieren ohne Genauigkeit hinzuzufügen. Abstürze\footnote{Als Absturz wird jeder Bodenkontakt mit mehr als $2,5\frac{m}{s}$ Restgeschwindigkeit gewertet.} werden grundsätzlich nicht als sinnvolle Trajektoren betrachtet.

Betrachten wir nun die Testreihe mit diesen Informationen im Hintergrund. Grundsätzlich steigt die maximale $g$-Kraft in Abhängigkeit von Eintrittsgeschwindigkeiten und Eintrittswinkel. Links unten sind die niedrigsten Werte, rechts oben die höchsten. Dies entspricht vermutlich auch der intuitiven Erwartung. Es zeigt sich allerdings ein lokales Maximum mit dem Zentrum bei $\alpha = 15^{\circ}$ und $v = 6000\frac{m}{s}$. Die Maximalwerte liegen bei $65g$, die Minimalwerte bei 13. Ein für diesen $L/D$-Wert typischer Trajektor ist der für die Werte $\alpha = 15^{\circ}$ und $v = 6000 \frac{m}{s}$, dargestellt in Abbildung \ref{fig:trajectory-0-15-6000}.
\centerImage{trajectory-0-15-6000}{0.35}{Flugbahn für  $L/D = 0$, $\alpha = 15^{\circ}$ und $v = 6000 \frac{m}{s}$}
Charakteristisch ist eine leichte Abflachung, die erst durch das Auslösen des Fallschirms deutlich verändert wird. Der dazugehörige Zeitpunkt ist auf dem Diagramm der g-Kräfte (Abbildung \ref{fig:g-0_15-6000}) gut erkennbar. Auffällig sind auf diesem Diagramm einige Artefakte der Modellierung. Zum einen der extreme Ausschlag bei 9 Sekunden, eine Folge der Singularität im Atmosphärenmodell (\ref{par:atmosphere}). Dann der zweite, bei dem die Beschleunigung von $\simeq1,6$ auf $112g$ springt. Dieser Sprung entsteht durch das die Modellierung der dynamischen Parameter.
\centerImage{g-0_15-6000}{0.35}{$g$-Kräfte für $L/D$ = 0, $\alpha = 15^{\circ}$ und $v = 6000 \frac{m}{s}$}
Die letzte Grafik für diesen Absatz (\ref{fig:multiTrajec0}) überlagert alle Bahnen für $L/D = 0$. Die zusammenhängenden Grüppchen teilen sich jeweils den selben Eintrittswinkel.
\centerImage{multiTrajec0}{0.35}{Alle Bahnen für $L/D = 0$}

\paragraph{$L/D = 0,1$}
Ab $L/D = 0,1$ verändern sich die Flugbahnen deutlich. Die sonst streng konvexen Kurven bekommen in nach Winkel und Geschwindigkeit variierender Höhe eine konkave Form. Die Extrempunkte Punkt bewegen auf einer Höhe von $60\pm 15km$. Bei kleinen Winkeln und hohen Ge\-schwin\-dig\-keit\-en beginnen sich ''Stufen'' zu bilden, bevor (in deutliche größerer Höhe als bei $L/D = 0$) der Fallschirm ausgelöst werden kann. Passend zu diesen Flugbahnen erhöhen sich die Flugzeiten und zurückgelegten Entfernungen. In Bezug auf die wirkenden Maximalbeschleunigungen bleibt die grundsätzliche Struktur erhalten. Die Fläche mit maximalen $g < 20$ verschiebt sich Richtung höherer Winkel und Geschwindigkeiten. Im Vergleich zu $L/D = 1$ bleibt nun die Kombination $\alpha = 22^{\circ}$ und $v = 6500$ unter $20g$. Gleichzeitig wird das lokale Maximum stärker. Der Wert für $\alpha = 15^{\circ}$ und $v = 6000 \frac{m}{s}$ steigt von $\simeq 37,3$ auf $\simeq 39,5$.

\centerImage{acc_0_1}{0.35}{$g$-Kräfte für $L/D$ = 0,1}
\centerImage{g-0_1-15-6000}{0.35}{$g$ für $L/D = 0,1$, $\alpha = 15^{\circ}$ und $v = 6000 \frac{m}{s}$}
\centerImage{multiTrajec0_1}{0.35}{Alle Bahnen für $L/D = 0,1$}


\paragraph{$L/D = 0,2$}
Der bei $L/D = 0,1$ beobachtete Trend setzt sich bei $L/D = 0,2$ in allen Bereichen deutlich fort. Waren für $L/D = 0,1$ erst ab Winkeln von kleiner als $25^{\circ}$ deutliche Abflachungen zu sehen, setzen diese nun schon bei $35^{\circ}$ ein. Während die minimalen zurückgelegten Entfernungen kaum schwanken, unterscheiden sich die maximalen Entfernungen um $34km$. Beim Schritt davor betrug diese Differenz $76km$. Das ist darauf zurückzuführen, dass sich nun annähernd oder zeitweise tatsächlich horizontale Flugbahnen auf einer Höhe von $7$ (hohe Winkel) bis $30km$ (kleine Winkel) ergeben. Das erlaubt es der Kapsel, länger Bewegungsenergie in Reibungskräfte umzuwandeln. Deshalb wird schneller die Grenze von Mach 2 unterschritten, ab der der Fallschirm ausgebracht werden kann.

Die Maximalbeschleunigungen entwickeln sich ebenfalls entlang der bisher beobachteten Richtung. Bei gleichem $\alpha = 22^{\circ}$ erhöht sich die mögliche Geschwindigkeit für $g > 20$ von $6055\frac{m}{s}$ auf $6217\frac{m}{s}$. Das lokale Maximum zieht sich deutlich zusammen, ohne jedoch bei diesem Schritt den Extremwert im Zentrum nennenswert zu erhöhen. Selbst für den Flug direkt im Zentrum des lokalen Maximums ($\alpha = 22^{\circ}, v = 6000\frac{m}{s}$) flacht sich die g-Kurve deutlich ab (\ref{fig:g-0_2-15-6000})

\centerImage{acc_0_2}{0.35}{$g$-Kräfte für $L/D = 0,2$}
\centerImage{g-0_2-15-6000}{0.35}{$g$ für $L/D = 0,2$, $\alpha = 15^{\circ}$ und $v = 6000 \frac{m}{s}$}
\centerImage{multiTrajec0_2}{0.35}{Alle Bahnen für $L/D = 0,2$}

\paragraph{$L/D = 0,3$}
Die Ergebnisse für $L/D$ sind noch stärker vom Winkel abhängig, als die bisherigen. Für steilere Winkel werden extreme Bahnen theoretisch möglich. Die ''extremste'' erfolgreiche Landung ist hierbei hier die mit den Werten $\alpha = 22^{\circ}$ und $v = 6000\frac{m}{s}$. Ob sie angesichts der extremen $g$-Entwicklung jedoch realistisch ist, bleibt fraglich. Bei unwesentlich größeren Winkeln beginnen die Bahnen, an der Atmosphäre abzuprallen. Die maximal zurückgelegte Entfernung beträgt hierbei $\simeq 1200km$. Das ist ca. ein Drittel des Mars-Radius. Bei dieser Entfernung dürfte die Krümmung der Oberfläche einen deutlichen Einfluss haben, wodurch sich Zweifel am Realismusgrad der Trajektoren für $10^{\circ}$ ergeben. Falls, wie zu vermuten, die Oberflächenkrümmung eine entscheidende Rolle spielt, müsste sich eine Form ergeben, bei der die Kapsel in mehrmals an der unteren Atmosphäre abprallt. Vergleichbar mit einem auf Wasser geflippten Stein würde sie bei jedem Eintritt Energie verlieren, bis die Geschwindigkeit zum Eintritt reicht. Solche Formen dürften extrem schwer mit Präzision zu planen sein. Unabhängig von der Qualität der Modelle gibt es eintretende unvorhersehbare Störfaktoren (insbesondere Wind), die nicht einkalkuliert werden können.

Das lokale Maximum bleibt erhalten. Die Erhöhung des $L/D$-Werts erlaubt es wieder, sich dessen Zentrum anzunähern, ohne die gewünschte g-Zahl zu überschreiten. Die Höhe des Zentrums verändert sich dabei für selbst für 6 Nachkommastellen nicht. Es entwickelte sich eine ausgeprägte Zunge mit relativ niedrigen $g$-Werten ($\leq 24,1$), deren Entwicklung sich über die gesamten Werte von $L/D$ ebenfalls gut verfolgen lässt.

\centerImage{acc_0_3}{0.35}{$g$-Kräfte für $L/D = 0,3$}
\centerImage{g-0_3-40-8000}{0.35}{$g$ für $L/D = 0,3$, $\alpha = 40^{\circ}$ und $v = 8000 \frac{m}{s}$}
\centerImage{multiTrajec0_3}{0.35}{Alle Bahnen für $L/D = 0,3$}