\label{subsec:tool}
Das Tool unterstützt die dynamischen Auswertung der Simulationsdaten. Es erlaubt einem, beliebige Simulationsdurchläufe auszuwählen. Für diese werden dann Kerngrößen des jeweiligen Landungsversuchs in scrollbaren Diagrammen aufbereitet. Als wahrscheinlich wichtigste Eigenschaft stellt es im Vergleich zu der Simulink Variante Diagramme mit normierten Achsen vor, was die Einordnung der vielen verschiedenen Kurven deutlich erleichtert. Auch werden wichtige Kennzahlen und Informationen dargestellt. Zu diesen Informationen gehört unter anderem der Ausgang des Flugs (Landung, Absturz, bei Simulationsende noch in der Luft), oder die letzte Geschwindigkeit.
\centerImage{tool}{0.16}{Screenshot des Analysetools}
Neben den Daten, die zu einzelnen Simulationsläufen gehören, generiert das Tool auch aggregierende Sichten auf die Daten. Allen voran war ein Ziel, eine aggregierende Sicht auf die wie erwähnt vierdimensionalen Daten Beschleunigung in $G$ in Abhängigkeit zu den Parametern $L/D$, $\alpha$ und $v$ (s. \ref{subsec:reihen}). Alle erstellten Diagramme können auch abgespeichert werden. Mit dieser Funktion wurden beinahe alle Diagramme dieser Arbeit erstellt. Das Tool benutzt die frei verfügbare OxyPlot \footnote{https://github.com/oxyplot/oxyplot} Bibliothek zum Erstellen der Grafiken. Deshalb ist das Tool in C\#/WPF erstellt. Der Sourcecode ist ebenfalls bei Veröffentlichung frei\footnote{https://github.com/jason-wilmans/mdp-mars-landing.git} verfügbar.