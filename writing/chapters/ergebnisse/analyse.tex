Wenn man die theoretisch möglichen Eintrittswinkel mit den Ausgängen anreichert, ergibt sich, dass von den betrachteten Variablen der Eintrittswinkel die wichtigste Größe ist. Steile Eintrittswinkel führen zu Abstürzen oder im Vergleich zu flacheren Winkeln ungleich höheren Beschleunigungs- und Hitzewerten. Zum Teil kann in diesem Bereich durch Materialforschung zumindest für Ladungen, die hohe G-Kräfte aushalten, noch Spielraum erarbeitet werden.

Betrachtet man die dadurch entstehenden Korridore für Eintrittswinkel als gegeben, fängt der Gleitwert an, eine entscheidende Rolle zu spielen. Im Vergleich zum Flugzeugbau minimale Verbesserungen des $L/D$-Werts eröffnen deutliche Spielräume im Vergleich zum freien Fall ohne Auftrieb. Als allgemeines Phänomen ist eine Verteilung der g-Kräfte über die Zeit zu beobachten. Das ist dadurch bedingt, dass die Zeit in der Atmosphäre maximiert werden kann. Der Einstieg in die tieferen Schichtend er Atmosphäre ist flacher, was die Druckgradienten positiv beeinflusst.

Vor Erstellung der Arbeit unerwartet war das Bestehen des lokalen Maximums der g-Kräfte unabhängig vom $L/D$-Wert. Bei der Recherche bewegten sich viele Ausgangspunkte in der Nähe dieses lokalen Maximums. Das legt nun die Vermutung nahe, dass dieses absichtlich ausgenutzt wird, um die gewünschte Bremswirkung zu dosieren, ohne sich den Risiken der anderen Extrema auszusetzen.