\documentclass[draft=false
              ,paper=a4
              ,twoside=false
              ,fontsize=11pt
              ,headsepline
              ,BCOR10mm
              ,DIV11
              ]{scrbook}
\usepackage[ngerman,english]{babel}
%% see http://www.tex.ac.uk/cgi-bin/texfaq2html?label=uselmfonts
\usepackage[T1]{fontenc}
%\usepackage[utf8]{inputenc}
%\usepackage[latin1]{inputenc}
\usepackage{libertine}
\usepackage{pifont}
\usepackage{microtype}
\usepackage{textcomp}
\usepackage[german,refpage]{nomencl}
\usepackage{setspace}
\usepackage{makeidx}
\usepackage{listings}
\usepackage{natbib}
\usepackage[ngerman,colorlinks=true]{hyperref}
\usepackage{soul}
\usepackage{hawstyle}
\usepackage{amsmath}
\usepackage{amssymb}
\usepackage{lipsum} %% for sample text
\usepackage{custom}

%% define some colors
\colorlet{BackgroundColor}{gray!20}
\colorlet{KeywordColor}{blue}
\colorlet{CommentColor}{black!60}
%% for tables
\colorlet{HeadColor}{gray!60}
\colorlet{Color1}{blue!10}
\colorlet{Color2}{white}

%% configure colors
\HAWifprinter{
  \colorlet{BackgroundColor}{gray!20}
  \colorlet{KeywordColor}{black}
  \colorlet{CommentColor}{gray}
  % for tables
  \colorlet{HeadColor}{gray!60}
  \colorlet{Color1}{gray!40}
  \colorlet{Color2}{white}
}{}
\lstset{%
  numbers=left,
  numberstyle=\tiny,
  stepnumber=1,
  numbersep=5pt,
  basicstyle=\ttfamily\small,
  keywordstyle=\color{KeywordColor}\bfseries,
  identifierstyle=\color{black},
  commentstyle=\color{CommentColor},
  backgroundcolor=\color{BackgroundColor},
  captionpos=b,
  fontadjust=true
}
\lstset{escapeinside={(*@}{@*)}, % used to enter latex code inside listings
        morekeywords={uint32_t, int32_t}
}
\ifpdfoutput{
  \hypersetup{bookmarksopen=false,bookmarksnumbered,linktocpage}
}{}

%% more fancy C++
\DeclareRobustCommand{\cxx}{C\raisebox{0.25ex}{{\scriptsize +\kern-0.25ex +}}}

\clubpenalty=10000
\widowpenalty=10000
\displaywidowpenalty=10000

% unknown hyphenations
\hyphenation{
}

%% recalculate text area
\typearea[current]{last}

\makeindex
\makenomenclature

\begin{document}
\selectlanguage{ngerman}

%%%%%
%% customize (see readme.pdf for supported values)
\HAWThesisProperties{Author={Moritz Mustermann}
                    ,Title={Softwareentwicklung im Gro�en und Ganzen	}
                    ,EnglishTitle={Developing software in Germany}
                    ,ThesisType={Bachelorarbeit}
                    ,ExaminationType={Bachelorpr�fung}
                    ,DegreeProgramme={Bachelor of Science Angewandte Informatik}
                    ,ThesisExperts={Prof. Dr. Erstpr�fer \and Prof. Dr. Zweitpr�fer}
                    ,ReleaseDate={1. Januar 2345}
                  }

%% title
\frontmatter

%% output title page
\maketitle

\onehalfspacing

%% add abstract pages
%% note: this is one command on multiple lines
\HAWAbstractPage
%% German abstract
{Schl�sselwort 1, Schl�sselwort 2}%
{Dieses Dokument \ldots}
%% English abstract
{keyword 1, keyword 2}%
{This document \ldots}

\newpage
\singlespacing

\tableofcontents
\newpage
%% enable if these lists should be shown on their own page
%%\listoftables
%%\listoffigures
\lstlistoflistings

%% main
\mainmatter
\onehalfspacing
%% write to the log/stdout
\typeout{===== File: chapter 1}

\section{Einleitung}

Motivation: Automatisierte Landungen auf Mars für Kolonie

Forschungsfrage
Ursprünglich: Treibstoffverbrauch für Raketenlandung
Aber sample size: 1, wenig Referenzen/Daten (insbesondere kein Triebvwerksverbrauch), Interaktion mit Atmosphäre komplexer
Die Marsatmosühäre ist dicht genug, um deutlichen Einfluss auf Trajektorien

Jetzt:
Einfluss von Drag auf Trajektorien (in Abhängigkeit zum Eintrittswinkel)
+ wirkende Kräfte (Realismus)

\section{Modellbeschreibung}
\subsection{Referenzmissionen}
	Die UdSSR 
Es gab schon erfolgreiche Missionen, die als Vorlage dienen. Hauptbezug: Curiosity/Mars Sciene Laboratory

\subsection{Vereinfachende Annahmen}
	Auf dem Mars herrschende Winde müssten auf (TODO: ignoriert)
	
\subsection{Übersicht über Modellteile}
	In diesem Abschnitt wird das Simulationsmodell vorgestellt. Das Modell wurde vollständig in MATLAB/Simulink\footnote{http://de.mathworks.com/products/simulink/} realisiert. Bei dieser Version des Modells handelt es sich um eine zweite Version. In der ersten wurden große Teile des Modells in so genannten MATLAB Functions\footnote{http://de.mathworks.com/help/simulink/slref/matlabfunction.html} ausgedrückt. Diese Vorgehensweise scheint nicht der bevorzugte Weg für MATLAB zu sein. Insbesondere beim automatisierten Linearisieren des Modells entstanden häufig Fehler. Die Erkenntnisse wurden in die zweite (hier vorgestellte) Version übertragen. Dieses Modell wurde von Anfang an als 2-D Version konzipiert. Insbesondere wurde versucht, wann immer möglich, Zusammenhänge über die von MATLAB angebotenen Signal Blocks auszudrücken.\\

Die Simulation ist in einer für Simulink typischen System-Subsystem-Struktur hierarchisch aufgebaut. Auf höchster Ebene unterscheiden sich die Flugsteuerung und das Flugmodell. Die Flugsteuerung hat zur Aufgabe, über die zur Verfügung stehenden Aktuatoren regelnden Einfluss auf den Flug zu nehmen. Hierzu überwacht es einige Kernparameter, wie die aktuelle Höhe.

Dem gegenüber steht das Flugmodell. Es modelliert die wirkenden physikalischen Kräfte und ihre Auswirkung auf wichtige Größen. Die Steuerbefehle der Flugsteuerung beeinflussen diese. Zusammen bilden die beiden Komponenten einen (indirekten) Regelkreis.
	
\subsection{Flugsteuerung}
	Die Flugsteuerung hat wiederum hat zwei erwähnenswerte Unterteilungen. Zum einen gibt es die Überwachung der Flugphasen. Diese orientiert sich stark an den Referenzmissionen. Sie ist intern als Deterministischer Endlicher Automat \centerImage{dea}{0.3}{Landephasen} modelliert. Im Unterschied zur Realität hat diese Phasenplanung keinerlei "Sicherheitsabstände" zwischen den Flugphasen. Stattdessen finden Übergänge ohne Zeitverzögerung statt (TODO: Evtl besser erklären?). Auch wurden alle Events ausgelassen, die das Gewicht oder den Schwerpunkt des Landesystems verändern.\\ \\

Der zweite Teil der Flugsteuerung ist das Controller-Setup. Sobald die "powered descent" Landephase beginnt, wird die verbleibenden Flughöhe als Signal in einen PID-Regler gespeist. In Reaktion auf dieses Signal bestimmt dieser Stärke der Triebwerke. Das Controller-Setup bildet während der "descent" Phase zusammen mit dem Flugmodell einen vollständigen Regelkreis.

	
\subsection{Flugmodell}
	Das Flugmodell berechnet die tatsächliche Physik des Fluges. Es besteht aus vier Teilsystemen, welche nun im Detail vorgestellt werden: Transform, Schwerkraft, Atmosphäreninteraktion und Triebwerke. \\ \\

\paragraph{Transform}
\label{transform}
Das Transform verwaltet die drei Variablen Geschwindigkeit, Position und Rotation. Das System hat als Parameter den aktuellen Beschleunigungsvektor. Die Beschleunigung entspricht der Summe der Einzelbeschleunigungen der anderen vier Blöcke. Die Geschwindigkeit ist das Integral der Beschleunigung, die Position integriert entsprechend die Geschwindigkeit.

Die Rotation hingegen ist nicht als unabhängige dynamische Größe modelliert. Sie wird als ideal gesteuert, i.e. immer der aktuellen Tangente des Flugtrajektors entgegengesetzt, angenommen. Entsprechend wird die Rotation als ein Vektor $\vec r$ dynamisch aus dem aktuellen Geschwindigkeitsvektor $\vec v$ berechnet als $\vec r = -\frac{\vec v}{|\vec v|}$.\\ \\

\paragraph{Gravitation}
Die Gravitation berechnet sich entsprechend der allgemeinen Formel für die Gravitationskraft.
$$F = G \frac{m_1 \cdot m_2}{r^{2}} $$
Sie berücksichtigt die Masse des Mars $m_M = 5.9724 \cdot 10^{24}$kg \cite{NASA2016}, die Masse der Landekapsel $m_K = 2401$kg \cite{Wikipedia2016b} und die dynamische Masse des Treibstoffs $m_T$, die zwischen 0 und 390 kg \cite{Wikipedia2016b} liegen kann. Mit $F = m \cdot a$ berechnet sich der Beschleunigungsvektors $\vec a_G$ wie folgt.
$$\vec a_G = G \frac{m_M \cdot (m_K + m_T)}{r^{2}} \cdot \frac{1}{m_K + m_T} \cdot \left(\begin{array}{c} 0 \\ -1 \end{array}\right)$$\\

\paragraph{Triebwerke}
Weitere Beschleunigung erfährt die Kapsel potentiell durch das Antriebssystem. Antriebssysteme definieren sich direkt durch ihre Schubkraft $F_T$. Diese ist laut Herstellerangaben zusammengerechnet 24.8 kN \cite{AerojetRocketdyne2012} \cite{AerojetRocketdyne}. Die auf Bildern angedeutete Neigung der verschiedenen Triebwerke am MSL ist hierbei abstrahiert. Auch die Positionierungen der Düsen (und damit die potentiell entstehenden Momente) wurden zugunsten einer niedrigeren Komplexität zu einem gebündelten Strahl, der im Schwerpunkt greift, vereinfacht. So ist die Beschleunigung mit dynamischem Gewicht  $ a_T = \dfrac{F_T}{m_K + m_T}$.

Allerdings muss die Stärke für die Flugsteuerung einstellbar sein. Um dies zu berücksichtigen, wird die Kraft mit dem Parameter $t \in \{t | t \in \mathbb{R} \land 0 \geq t \leq 1\}$ multipliziert. Um den Realismus des Triebwerkes deutlich zu erhöhen, wird der Parameter $t$ allerdings nicht direkt benutzt. Das Signal der Flugsteuerung wird stattdessen um 200 ms verzögert und simuliert mit Hilfe einer Transfer Function (TODO: Parametrisierung, außerdem: erklären? ) ein Sättigungsverhalten. Notiert man die Verzögerung $d$(elay) und die Sättigung $s$(aturation) als Funktionen auf der gewünschten Leistung, ergibt sich die Gleichung des Beschleunigungsvektors. $\vec r$ ist dabei der unter \ref{transform} beschriebene, normalisierte, Richtungsvektor der Kapsel.
$$\vec a_{T} = \frac{d(s(t))F_T}{m_K + m_T} \cdot \vec r$$\\

\paragraph{Atmosphreninteraktion}
Der komplexeste Teil des Modells beschreibt die Interaktion mit der Atmosphäre. Es werden zwei resultierende Kräfte berechnet. Die erste ist der Luftwiderstand, dem die Kapsel immer ausgesetzt ist. Zusätzlich kann, je nach Form und Angriffswinkel, dynamischer Auftrieb erzeugt werden. Die Berechnung beider Kräfte basiert auf einem Atmosphärenmodell, dass den Luftdruck in Abhängigkeit von der Höhe annähert. Ebenfalls großen Einfluss (und im Sinne der Fragestellung hoch relevant) haben die Flugparameter, die sich je nach Flugphase stark ändern können.

\paragraph{Kräfte}
Auf die detaillierte Berechnung der Strömungsarten wurde auf Grund der schlechten Quellenlage und der praktisch unendlich steigerbaren Komplexität des Themas verzichtet. Die zugrunde liegende Komplexität drückt sich allerdings abgeschwächt in den dynamischen Parametern aus. Die Berechnung der tatsächlich entstehenden Kräfte verhält sich analog zu den bisher betrachteten Teilsystemen. Es gelten die allgemeinen Gleichungen für Luftwiderstand
$$F_W = c_W \cdot \frac{p}{2} \cdot v^_{2} \cdot A $$
und Auftrieb:
$$F_A = c_A \cdot \frac{p}{2} \cdot v^_{2} \cdot A $$

Die korrespondierenden Beschleunigungsvektoren $\vec a_W$ und $\vec a_A$ berechnen sich

		NASA Atmosphärenmodell (Anpassungen, Singularität!)
			Luftwiderstand
			Auftriebskraft
		dynamische Parameter

\section{Experimente}
%\subsection{Versuchsreihen}
Variables:
L/D = 0.0 to 0.3 (step size: 0.05 = 5 steps)
Entry Speed = 3000 to 9000 m/s \cite{Edquist2009} (Stepsize 1000 m/s = 6 steps)
Entry Angle = -10° to 20° (step size: 2° = 5 steps)

Results:
Time
Trajectory
Max/Min Air Speed and Speed on Impact
Total Horizontal Distance
Max Force

\section{Ausblick}
%\label{sec:ausblick}Das Modell könnte an vielen Stellen verbessert und ausgebaut werden. Das Auslassen der Oberflächenkrümung des Mars schränkt das besonderes sowohl für die Spanne der Gleitwerte, als auch für die Eintrittswinkel ein. Für diese Arbeit wurde der $L/D$ als konstant angenommen. Durch das Ändern des Angriffswinkels der Kapsel kann dieser allerdings variiert werden. Es bietet sich an, die Zugewinne durch einen dynamisch steuerbaren Auftrieb zu untersuchen. Dazu müsste das Modell um ''echte'' Rotation erweitert werden. Dadurch eröffnete sich auch die Möglichkeit, eine größere Spannweite von Fahrzeugen abzubilden. Besonders interessant wären Raketen mit der Möglichkeit, zur Erde zurück zu kehren.

Als etwas kleinere Weiterentwicklungen kommen unter anderem die Annahmen zum Übergang der Landephasen in Betracht. Auch könnte die dynamische Flächenberechnung deutlich genauer aufgelöst werden. Das würde die Abbildung anderer Objekte unterstützen. Als letzter großer Punkt bleibt das System der dynamischen Parameter. Es liefert zwar eine Annäherung an die echten Verhältnisse. Besser wäre aber eine dynamische Berechnung statt des tabellarischen Abspulens fester Werte.

\section{Fazit}
%Um die eingehend beschriebenen Forschungsfragen zu beantworten, wurde ein Modell implementiert und dessen Ergebnisse untersucht. Es hat sich gezeigt, dass dieses Modell - obwohl von der Realität noch weit entfernt - innerhalb gewisser Grenzen annähernd realistische Ergebnisse liefert. Es bildet einen guten Grundstein für potentielle weiter Untersuchungen.

Insbesondere können die Forschungsfragen wie folgt beantwortet werden: Der Gleitfaktor hat nach dem Eintrittswinkel den größten Einfluss auf die realistischen Trajektorien. Insbesondere kann eine gleitende Kapsel die auftretenden Beschleunigungskräfte und damit Hitzeentwicklung gleichmäßiger verteilen. Eine erhöhte $L/D$ erlaubt es, den Bereich des Möglichen für Faktoren wie die Geschwindigkeit oder den Eintrittswinkel zu vergrößern.

%%%%

%% appendix if used
%%\appendix
%%\typeout{===== File: appendix}
%%\include{appendix}

% bibliography and other stuff
\backmatter

\typeout{===== Section: literature}
%% read the documentation for customizing the style
\bibliographystyle{dinat}
\bibliography{library}

\typeout{===== Section: nomenclature}
%% uncomment if a TOC entry is needed
%%\addcontentsline{toc}{chapter}{Glossar}
\renewcommand{\nomname}{Glossar}
\clearpage
\markboth{\nomname}{\nomname} %% see nomencl doc, page 9, section 4.1
\printnomenclature

%% index
\typeout{===== Section: index}
\printindex

\HAWasurency

\end{document}
